\documentclass{beamer}
\usetheme{Warsaw}
\usepackage{graphicx}
\usepackage{listings}
\usepackage[utf8]{inputenc}

\title{A load of stylistic advice for technical writing}

\author{Giuseppe Maggiore}

\institute{NHTV University of Applied Sciences \\ 
Breda, Netherlands}

\date{7th January 2013}

\begin{document}
\maketitle

\begin{frame}{Agenda}
\tableofcontents
\end{frame}

\section{Good writing}
\begin{frame}{Basics}
\begin{block}{Good writing and good ideas}
\begin{itemize}
\item good writing is essential in a dissertation, but it cannot compensate for paucity of ideas or concepts
\item a clear presentation actually exposes weaknesses
\end{itemize}
\end{block}
\end{frame}

\begin{frame}{Basics}
\begin{block}{Clarity}
\begin{itemize}
\item each sentence in a dissertation must be complete and correct in a grammatical sense
\item writing must be crystal clear
\end{itemize}
\end{block}
\end{frame}

\begin{frame}{Basics}
\begin{block}{Clarity}
\begin{itemize}
\item shades of meaning matter: terminology and prose must make fine distinctions
\item the words must convey \textit{exactly} the meaning intended
\end{itemize}
\end{block}
\end{frame}

\begin{frame}{Basics}
\begin{block}{Pitfalls}
a dissertation must satisfy the stringent rules of formal grammar, and avoid:
\begin{itemize}
\item contractions, such as ``not'' for ``-n't'', ``have'' for ``-'ve''
\item colloquialisms, such as ``old as the hills'', ``raining cats and dogs'', ``dead as a doornail'', etc.
\item slang, ``cool'' and ``hot''
%\item slurs, ``kids'', ``old fart''
\item undefined technical jargon, 
\item hidden jokes
\end{itemize}
\end{block}
\end{frame}

\begin{frame}{Basics}
\begin{block}{Technical terms}
\begin{itemize}
\item each technical term used in a dissertation must be defined:
  \begin{itemize}
  \item a reference for standard terms
  \item a precise, unambiguous definition that appears before the term 
  \end{itemize}
\item each term should be used in one and only one way throughout the dissertation
\item the introduction can give the intuition of terms, provided a precise definition is given later
\end{itemize}
\end{block}
\end{frame}

\begin{frame}{Grammar and style}
\begin{block}{Style}
\begin{itemize}
\item avoid adverbs: ``mostly, they are very often overly used''
\item use strong words: ``writers abuse adverbs''
\end{itemize}
\end{block}
\end{frame}

\begin{frame}{Grammar and style}
\begin{block}{Style}
\begin{itemize}
\item use the active voice: ``the operating system starts the device'', not ``the device is started by the operating system''
\item avoid future tense: ``the system writes a page to the disk and then uses the frame ...'' instead of ``the system will use the frame after it wrote the page to disk ...''
\end{itemize}
\end{block}
\end{frame}

\begin{frame}{Grammar and style}
\begin{block}{Sentence structure}
\begin{itemize}
\item negation early: ``no data block waits on the output queue'' instead of ``a data block awaiting output is not on the queue''
\item be careful that the subject really does what the verb says: ``RPC requires programs to transmit large packets'' is not the same as ``RPC requires a mechanism that allows programs to transmit large packets''
\end{itemize}
\end{block}
\end{frame}

\begin{frame}{Grammar and style}
\begin{block}{First person}
\begin{itemize}
\item the author is invisible, and talks from the perspective of the done work: ``we were surprised to learn'' is never said
\item no first person: ``I will describe'' should be ``Section 7 describes''
\end{itemize}
\end{block}
\end{frame}

\begin{frame}{Moral judgements}
\begin{block}{Amoral assessment}
\begin{itemize}
\item no moral judgements such as ``true, pure, bad, good, nice, terrible, stupid''; things are ``correct/incorrect''
\item quality assessments are done precisely: ``method A requires less computation time than method B''
\end{itemize}
\end{block}
\end{frame}

\begin{frame}{Moral judgements}
\begin{block}{No qualitative judgements}
\begin{itemize}
\item avoid qualitative judgements such as ``better/worse''
\item ``obvious, clear, simple'' is vague and often slightly insulting
\item nothing is ``perfect'', or ``an ideal solution''
\end{itemize}
\end{block}
\end{frame}

\begin{frame}{Moral judgements}
\begin{block}{Minor mistakes}
\begin{itemize}
\item ``a famous researchers'' is never relevant; also, do not prejudice the reader
\item ``commercial success'' is never relevant; also, do not prejudice the reader
\item ``hopefully'': hope has no place in science
\item ``should'': who says so?
\end{itemize}
\end{block}
\end{frame}

\begin{frame}{Precision}
\begin{block}{Minor mistakes}
Precision is of paramount importance in science:
\begin{itemize}
\item ``few, most, all, any, every'' are not precise enough
\item ``all, any, every'' are almost always false
\end{itemize}
\end{block}
\end{frame}

\begin{frame}{Precision}
\begin{block}{Minor mistakes}
Precision is of paramount importance in science:
\begin{itemize}
\item ``most computer systems contain X'': are you sure you really know the facts? How many computers were built and sold yesterday? 
\item For example, did you know that books sell more than films or music \textit{combined}?
\end{itemize}
\end{block}
\end{frame}

\begin{frame}{Precision}
\begin{block}{Minor mistakes}
\begin{itemize}
\item ``today, modern times'' are tomorrow's yesterday
\item ``soon'' means nothing: it is true now, or it is irrelevant
\end{itemize}
\end{block}
\end{frame}

\begin{frame}{Precision}
\begin{block}{Minor mistakes}
\begin{itemize}
\item ``seems, seemingly, would seem to show'': appearances do not matter; facts and substance do
\item ...unless you are making a point to ensure the reader does not fall into an intuitive trap
\end{itemize}
\end{block}
\end{frame}

\begin{frame}{Precision}
\begin{block}{Minor mistakes}
\begin{itemize}
\item ``in terms of, based on'' is usually too vague
\item ``in light of'' is too colloquial
\end{itemize}
\end{block}
\end{frame}

\begin{frame}{Precision}
\begin{block}{Minor mistakes}
\begin{itemize}
\item ``lots of, kind of, type of, just about, something like, number of, due to, probably'' are too vague and colloquial
\end{itemize}
\end{block}
\end{frame}

\begin{frame}{Precision}
\begin{block}{Risky constructs}
\begin{itemize}
\item ``actually, really'' often imply imprecision; restructure so as to avoid the need for clarification
\end{itemize}
\end{block}
\end{frame}

\begin{frame}{Precision}
\begin{block}{Risky constructs}
\begin{itemize}
\item ``this, that'' can refer to too many things: ``X does Y. \underline{This} means ...'' \underline{Y}? \underline{does}?
\item ``proof, prove, show'' would a mathematician agree? (show $=$ prove)
\item ``can/may''; ``the algorithm can compute the $F$ of $X$ up to ten thousand elements ...''; ``a user may choose option A or option B''
\end{itemize}
\end{block}
\end{frame}

\section{Principles and practices}
\begin{frame}{Principles to adhere to}
\begin{block}{Core principles of correct scientific writing}
\begin{itemize}
\item \textbf{Correctness.} Write correct English, but know that you have more latitude than your high-school English teachers may have given you.
\item \textbf{Consistent names.} Refer to each significant character (algorithm, concept, language) using the same word everywhere. Give a significant new character a proper name.
\item \textbf{Singular.} To distinguish \texttt{one-to-one} relationships from \texttt{n-to-m} relationships, refer to each item in the singular, not the plural.
\end{itemize}
\end{block}
\end{frame}


\begin{frame}{Principles to adhere to}
\begin{block}{Core principles of correct scientific writing}
\begin{itemize}
\item \textbf{Subjects and verbs.} Important characters in subjects; join each subject to a verb that expresses a significant action.
\item \textbf{Information flow.} In each sentence, move from familiar information to new information.
\item \textbf{Emphasis.} For material you want to carry weight or be remembered, use the end of a sentence.
\item \textbf{Coherence.} In a coherent passage, choose subjects that refer to a consistent set of related concepts.
\item \textbf{Parallel structure.} Order your text so your reader can easily see similar/different concepts.
\end{itemize}
\end{block}
\end{frame}


\begin{frame}{Practices to follow}
\begin{block}{Behaviours that lead to better texts}
\begin{itemize}
\item Write in brief daily sessions. Ignore the common myth that successful writing requires large, uninterrupted blocks of time.
\item Focus on the process, not the product. Do not worry about the size or quality of your output; instead, reward yourself for the consistency and regularity of your input.
\end{itemize}
\end{block}
\end{frame}


\begin{frame}{Practices to follow}
\begin{block}{Behaviours that lead to better texts}
\begin{itemize}
\item Prewrite. Do not be afraid to think before you write, or even jot down notes, diagrams, and so on.
\item Use index cards. Use them to plan a draft or to organize or reorganize a large unit like a section or chapter.
\item Write a \textit{Shitty First Draft \texttrademark}. Value a first draft not because it is great but because it is there.
\item Do not worry about page limits. Write the paper you want, then cut it down to size.
\end{itemize}
\end{block}
\end{frame}


\begin{frame}{Guidelines}
\begin{block}{Supplementary useful guidelines}
When explaining new concepts in science and engineering:
\begin{itemize}
\item enumerate all the properties of the thing
\item say whether the thing is completely characterized by those properties
\item give a name or symbol to each property
\item give the type/kind of each property
\item explain relationships that hold among the properties, and what forces them to hold
\end{itemize}
\end{block}
\end{frame}


\begin{frame}{Guidelines}
\begin{block}{Supplementary useful guidelines/2}
Examples help a lot when explaining complex concepts.
\begin{itemize}
\item Do you have examples? They are helpful, and they should
\begin{itemize}
\item Be plentiful 
\item Use parallel structure 
\item Be connected to each other when possible
\end{itemize}
Ideally a single running example appears in each section of the manuscript (supplemented by additional examples.)
\item Every general, abstract declaration is illustrated by an example
\end{itemize}
\end{block}
\end{frame}


\begin{frame}{Guidelines}
\begin{block}{Supplementary useful guidelines/3}
When presenting complicated technical abstractions:
\begin{itemize}
\item You may well have a nest of interrelated concepts 
\item Sometimes there is no obvious order of presentation
\item Make simplifications for pedagogical purposes (and announce them)
\item Mention a concept without defining it (``Let’s assume that \texttt{l} is a location on the stack, without going into the details, which are in Section \textit{12}.'')
\end{itemize}
\end{block}
\end{frame}


\begin{frame}{Guidelines}
\begin{block}{Supplementary useful guidelines/4}
When presenting complicated technical abstractions:
\begin{itemize}
\item Types help. Give the type of every operation
\item Explain the name of each variable or Greek letter; for example, explain that $\Gamma$ stands for a parsing environment
\end{itemize}
\end{block}
\end{frame}

\section{Research question or problem}
\begin{frame}{Problem statement}
\begin{block}{Finding the problem}
\begin{itemize}
\item a thesis identifies a \textit{worthwhile} problem or question
\item a (thorough) review of the existing literature on the subject shows your question to be original and valuable
\end{itemize}
\end{block}
\end{frame}


\section{Research question or problem}
\begin{frame}{Problem statement}
\begin{block}{Highlighting the problem statement}
\begin{itemize}
\item the research question should be stated \textit{as clearly as possible}
\end{itemize}
\end{block}
\end{frame}


\section{Scientific frame}
\begin{frame}{Thesis goals}
\begin{block}{Central concerns}
\begin{itemize}
\item a thesis is a hypothesis or conjecture
\item two important adjectives are ``original'' and ``substantial''
\end{itemize}
\end{block}
\end{frame}


\begin{frame}{Scientific methodology}
\begin{block}{Methodology}
\begin{itemize}
\item describe your scientific methodology, and why you have chosen it
\end{itemize}
\end{block}
\end{frame}


\begin{frame}{Scientific methodology}
\begin{block}{Process}
\begin{itemize}
\item start with a hypothesis, and then collect evidence to support or deny it
\item the most difficult part of a dissertation is the organization of the evidence and associated discussions into a coherent form
\end{itemize}
\end{block}
\end{frame}


\begin{frame}{Scientific methodology}
\begin{block}{Methodology}
\begin{itemize}
\item the facts that result from an experiment are called ``data'', the useful information condensed from the data is the resulting ``knowledge''
\item the aim of a thesis is always knowledge
\end{itemize}
\end{block}
\end{frame}


\begin{frame}{Scientific methodology}
\begin{block}{Goal}
\begin{itemize}
\item a dissertation concentrates on principles: it states the lessons learned, and not merely the facts behind them
\item this requires an effort in generalization; \textit{what useful bit of understanding comes from the results?}
\end{itemize}
\end{block}
\end{frame}


\begin{frame}{Scientific methodology}
\begin{block}{Methodology}
\begin{itemize}
\item the essence of a dissertation is \textit{critical thinking}, not experimental data
\item analysis and concepts form the heart of the work
\end{itemize}
\end{block}
\end{frame}


\begin{frame}{Dangers}
\begin{block}{Challenges to proper science}
\begin{itemize}
\item separate cause-effect relationships from statistical correlation
\item ``pirates absorb C02''
\item \url{http://www.forbes.com/sites/erikaandersen/2012/03/23/true-fact-the-lack-of-pirates-is-causing-global-warming/}
\end{itemize}
\end{block}
\end{frame}


\begin{frame}{Dangers}
\begin{block}{Challenges to proper science}
\begin{itemize}
\item only draw warranted conclusions: ``the same game on an XBox One is faster than on a PS4, therefore the PS4 is worse than the XBox One''
\item may depend on implementation, language, runtime, OS, etc.
\end{itemize}
\end{block}
\end{frame}


\begin{frame}{Dangers}
\begin{block}{Challenges to proper science}
\begin{itemize}
\item even if the cause of some phenomenon seems obvious, one cannot draw a conclusion without solid, supporting evidence
\item do not hide reliability problems with your testing methodology; your resources are constrained, so random samples or extensive tests may not be feasible
\item reasonable choices and honesty are always appreciated
\end{itemize}
\end{block}
\end{frame}

\begin{frame}{Results, results, results!}
\begin{block}{Strict exposition of results}
\begin{itemize}
\item focus on results, not people or circumstances
\item ''after eight hours of tests, Jim and I discovered the results now in Table 3''
\item ''if that cat had not crawled through the hole in the floor, we might not have discovered the power supply error indicator''
\end{itemize}
\end{block}
\end{frame}


\begin{frame}{Results, results, results!}
\begin{block}{Strict exposition of results}
\begin{itemize}
\item avoid self-assessment, such as ``we present a major breakthrough in the design of distributed systems'', or ``although the technique in the next section is not earthshaking, ...''
\end{itemize}
\end{block}
\end{frame}


\begin{frame}{Results, results, results!}
\begin{block}{Each statement works well}
\begin{itemize}
\item each statement in a dissertation must satisfy the most stringent rules of logic applied to mathematics and science
\end{itemize}
\end{block}
\end{frame}


\begin{frame}{Results, results, results!}
\begin{block}{Principles of knowledge}
\begin{itemize}
\item have a basis for all statements; either original work, or a reference to scientific literature
\item references should only be interested in others' results, not their methodology or analysis
\end{itemize}
\end{block}
\end{frame}


\begin{frame}{Results, results, results!}
\begin{block}{Sociality and science}
\begin{itemize}
\item never draw conclusions about economic viability or commercial success; ``over four hundred vendors make products using technique Y'' is irrelevant in a dissertation
\item it does not matter whether government bodies, political parties, religious groups, or other organizations endorse an idea
\item it does not matter who discovered something: a first year graduate student or a Nobel-prize winner
\end{itemize}
\end{block}
\end{frame}


\section{Other people's work}
\begin{frame}{Comparisons}
\begin{block}{On the shoulders of giants}
\begin{itemize}
\item a thesis should contain a chapter about what other people have done in the same or similar areas
\end{itemize}
\end{block}
\end{frame}


\begin{frame}{Comparisons}
\begin{block}{On the shoulders of giants}
\begin{itemize}
\item some web pages and wikipedia articles contain very scientifically sound information. Refer to them, possibly through \textit{WebCite}
\end{itemize}
\end{block}
\end{frame}


\begin{frame}{Comparisons}
\begin{block}{On the shoulders of giants}
\begin{itemize}
\item all references \textit{must} be referred to in the main body of the thesis
\item do not use footnotes for references
\end{itemize}
\end{block}
\end{frame}


\begin{frame}{Comparisons}
\begin{block}{Problem statement references}
\begin{itemize}
\item you should \textit{demonstrate} that the question has not been answered already and that it is worth answering
\item interestingly, it is easier to answer the question than to discuss it, as you will become familiar with the details of the solution by working on it
\item the question is harder to reason about since it stems from your intuition
\end{itemize}
\end{block}
\end{frame}


\begin{frame}{Comparisons}
\begin{block}{No plagiarism}
\begin{itemize}
\item avoid plagiarism at all costs; clearly specify every text you have quoted from someone else, and never use other people's text without indicating the source
\item it is \textit{exactly} a form of stealing; you are stealing someone else's hard work and using it to achieve your own personal results
\end{itemize}
\end{block}
\end{frame}


\section{Structure of the text}
\begin{frame}{Funnel}
\begin{block}{Funnel structure}
\begin{itemize}
\item introduction (broadest, intuition)
\item problem (still broad, precise)
\item solution (narrow, specific to the problem, precise)
\item assessment (narrowest, specific to the solution, very precise, analytic/synthetic)
\end{itemize}
\end{block}
\end{frame}


\begin{frame}{Writing strategy}
\begin{block}{From the intuition to the details}
\begin{itemize}
\item prepare an extended outline, one or two sentences for each chapter/section
\end{itemize}
\end{block}
\end{frame}


\begin{frame}{Writing strategy}
\begin{block}{From the intuition to the details}
\begin{itemize}
\item write the problem statement and assessment methodology first, then go to your supervisor
\item without these, he cannot determine if your thesis will work or not!
\end{itemize}
\end{block}
\end{frame}


\section{Common issues}
\begin{frame}{Broader issues}
\begin{block}{Not all theses are very good}
Many theses are accepted even though they are not very good because:
\begin{itemize}
\item they are too long
\item there is too much well-known information on the field, such as tutorials or user guides
\item they are too short
\item they contain detailed transcripts of data
\item they leave you wondering ``so what?''
\end{itemize}
\end{block}
\end{frame}


\begin{frame}{Remember your audience}
\begin{block}{Your thesis is read by people}
Answer the following supervisor questions with your thesis:
\begin{itemize}
\item what is the student's \textit{research question}?
\item is it a good (\textit{open}, \textit{novel}, \textit{relevant}) question?
\item did the student convince me that the question was \textit{adequately answered}?
\end{itemize}
\end{block}
\end{frame}


\section{In closing...}
\begin{frame}{Clarity and knowledge}
\begin{block}{Just remember}
\begin{itemize}
\item it is impossible to be too clear, so spell things out carefully
\item tools such as computer programs are fine and useful products, but the most important item of all is the ideas embodied by the tool, not the tool itself
\item remember that the purpose of your thesis is \textit{knowledge}
\end{itemize}
\end{block}
\end{frame}


\begin{frame}{This is it}
\begin{block}{ }
Thank you!
\end{block}
\end{frame}


%\begin{frame}[allowframebreaks]{Bibliography}
%\bibliographystyle{plain}
%\bibliography{bibliography}
%\end{frame}

\end{document}