\documentclass{article}

\usepackage{listings}
\usepackage[utf8]{inputenc}
\usepackage[T1]{fontenc} 
\usepackage[hyphens]{url}

\begin{document}

\textbf{Course name:} Writing skills 2: research

\textbf{Course code:} FGA3.WS2-02 (WS2)

\textbf{Academic year:} 2013-2014

\textbf{Lecturer:} Giuseppe Maggiore

\textbf{Number of EC's:} 2


\section{Introduction}
In this document we describe the \textit{Writing skills 2: research writing} course. The course aims at teaching students how to write good research documents. The objective is for students to be able to competently and clearly relate results obtained through some work-related activity, and correlating these results with results obtained by others and found through a review of the literature.

At the end of the course the students will be able to understand the structure of a paper written by others and to write their own, either for publication or for internal team use.

The course will discuss scientific papers, and how they are written. An initial review of the literature will allow the students to experiment with some research (limited in scope and size) of which they will then have to write a short paper. After getting confidence with the research process, the students will write a larger document which will act as the foundation of their specialization proposal.


\subsection{Relation to other courses}
This course relates to WS1 in Year 1 (also taught in the ‘ENG’ series of English courses) and contributes towards successful completion of both the specialization and graduation phases of the overall degree. The course also relates to FGA3.AGT6, in that the research documents for the first assignment will be about the work done for AGT6.


\subsection{Relation to the industry}
Game developers work in an ever-changing environment, where new techniques are quickly discovered and integrated into new products. A developer should be able to find, read, and learn from journal and conference proceedings in order to maintain his knowledge of the cutting edge techniques of his field.  Moreover, the ability to clearly express and explain one's technical ideas is crucial for communication, both in the form of internal documentation and for publication.

This course will give students experience in finding and using academic resources, and using them in an applied context. The course will also focus on improving the formal writing skills of the students.


\subsection{Course description}
The purpose of this course is to prepare the students for the Specialization and Graduation phases of their studies at NHTV. For the programming students this requires that the students get an oversight of methods of studying games and game development, and that they then learn ways to express those results in formally constructed documents.


\subsection{Competences}
For the full table of competences, see Table \ref{Table:Competences}. This course also relates, but less directly, to competency G2 (Documentation) where the use of formal English and clear structure are necessary for success.


\begin{table}[ht]
\caption{Competences}
\centering
\begin{tabular}{| p{4cm} | p{7cm} |}

\hline

\textbf{Competency} & \textbf{G8. Knowledge Updating} \\

\hline

Core problem and critical professional situation &

The game industry is characterized by constant changes in technologies and market developments, so a game developer needs to pro-actively keep their knowledge and skills up-to-date. Awareness of research methodologies is therefore imperative.

Digital media studies, technical research studies and innovations in games are critical to this competency. \\


\hline

Professional product &


An up-to-date knowledge and skills database. 
Critical analysis of existing and future trends. \\


\hline

Criteria regarding product and process (also referred to as indicators) &

Criteria regarding the process: The game developer pro-actively updates his/her knowledge. 

Criteria regarding the product: Knowledge and skills of the game developer are always up-to-date.  \\


\hline

Knowledge &

Knowledge of sources for relevant and appropriate information to keep up to date about trends in the games industry.

Knowledge of research methodologies \\


\hline

Skills \& Attitude &

Capable of directing one's own learning process.

Independent and proactive.

Research skills.

Critical analysis skills. \\

\hline

\end{tabular}
\label{Table:Competences}
\end{table}


\subsection{Course learning objectives}
The learning objectives of this course are:

\begin{itemize}
\item Students will demonstrate research skills by locating academic resources in their field of study
\item Students will demonstrate critical thinking through analysis and comparison of academic resources
\item Students will gain deeper knowledge of quantitative analysis and methods of practical research
\item Students will be able to convey their ideas and knowledge in formal written English.
\end{itemize}


\section{Course structure}
\subsection{Number of hours}
1.5 hours per week, contact time, and approximately 60 hours total study time.


\subsection{Attendance policy}
Attendance is not mandatory but students may miss valuable information if they do not attend the classes.


\subsection{Teaching method}
A mixture of lectures and practical tutorial sessions where the students discuss with the lecturer how to complete or improve their assignments.


\section{Assessment \& deadlines}
The course has two assessments. Both are to be delivered in print in the lecturer pigeon-hole. One assessment is the \textit{literature review}, while the second is a \textit{specialization proposal}.

The literature review is due Friday of week 5, block B, at 13:00. It amounts to 80\% of the final mark. The literature review must be between a 1,000 and 2,000 words, and no extra leeway will be granted. The literature review needs to present some existing work, from either: \textit{(i)} the main assignment of the AGT6 course; \textit{(ii)} an external project about game engine architectures; \textit{(iii)} an external project about functional programming for games.
A revision of the literature review that integrates the lecturer's feedback is due by the end of exam week. It amounts to 80\% of the final grade. When delivering your literature review revision, make sure to hand back in the original as annotated by the lecturer.

The specialization proposal is due by the end of exam week. It amounts to the remaining 20\% of the final mark. The specialization proposal should be between 400 and 600 words.

\subsection{Literature review}
The literature review consists of two main elements. Primarily it is a short paper that sums up a practical experience in solving an applied problem. Secondly, solving the problem satisfactorily requires researching the literature of game development, as published in industry or academic conferences, and experimenting with it in order to gather data. The short paper must be rigorous, must clearly state the goal of the students work, and it must back its conclusion with data.

The grading criteria will be:
\begin{itemize}
\item the clear statement of the problem (20\%)
\item the quantitative support of solution (20\%)
\item the quality and completeness of the references (20\%)
\item academic structure to the document (20\%)
\item the clarity and formality of writing (20\%)
\end{itemize}

You will be given a chance to revise your paper in order to integrate the lecturer's feedback, and possibly increase the initial grade. 

If the first paper that is handed in is judged inadequate, you may be denied this opportunity. This may happen in case the structure of the paper is excessively jumbled up, if the writing is too hard to understand overall, or in general in case of extensive and fundamental issues.
 

\subsection{Specialization proposal}
The specialization proposal will form the starting point of the students specialization studies. It will identify the key area that the student wishes to research and name the primary sources or methods that the student wishes to use, as well as the likely outcome (thesis, short thesis and practical assignment, or developed practical assignment).

The grading criteria will be:
\begin{itemize}
\item the clear statement of the problem (20\%)
\item a plan on how to achieve the final result (20\%)
\item the quality of the initial references (20\%)
\item academic structure to the document (20\%)
\item the clarity and formality of writing (20\%)
\end{itemize}


\section{Materials}
\subsection{Literature}
Students will be finding source materials from the library and online resources that are related to their field of study. Some example documents will be provided on Natschool.


\subsection{Software packages}
Latex and Bibtex.


\section{Weekly schedule}
\begin{itemize}
\item \textbf{week 1 - course introduction}
\begin{itemize}
\item 1 hour: course intro, description of specialization options 
\item 1 hour: course assessments, discussion of specialization topics
\end{itemize}
For next week: find 10 articles on the subject that interests the student and that may guide him towards a specialization topic; find and publish the preferred article on the subject of the assessment (planning and AI).

\item \textbf{week 2 - the scientific process}
\begin{itemize}
\item what is science?
\item what is research?
\item data analysis
\item references
\item differences between essay, paper, thesis
\end{itemize}

\item \textbf{week 3 - paper templates}
\begin{itemize}
\item heading and front matter
\item abstract
\item introduction
\item case study
\item evaluation; using good data, and not bad data
\item discussion
\item conclusions
\end{itemize}

\item \textbf{week 4 - tools}
Latex: type setting and bibliography
For next week: publish logs coming from the implementation of the AI.

\item \textbf{week 5 - working class}


\item \textbf{week 6 - common mistakes and additional working class} 
\begin{itemize}
\item judging a paper by format, methods used, and results obtained
\item proper writing: avoiding \textit{non scientific language}
\item clearly stating a problem
\item proper formalization
\item research requires \textit{general and repeatable} results
\end{itemize}

\item \textbf{week 7 - working class} Present, discuss, and review specialization proposals.

\end{itemize}

\section{Conclusions}
In this document we described the \textit{Writing skills 2: research} course. The course focuses on the writing of research documents through proper research methodology.

At the end of the course the students will be able to understand the structure of a paper written by others, and to write their own, both for publication and for internal team use.

\end{document}
